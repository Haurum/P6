\chapter*{Preface}


{\Huge\textbf{Signatures}}
\newline
\newline


\begin{table}[H]
	\centering
	\begin{tabular}{c c}
		\underline{\phantom{mmmmmmmmmmmmmm}} & \underline{\phantom{mmmmmmmmmmmmmm}} \\
		Emil Bønnerup			& Frederik Børsting Lund	\\
		&\\
		&\\
		\underline{\phantom{mmmmmmmmmmmmmm}} & \underline{\phantom{mmmmmmmmmmmmmm}} \\
		Mark Kloch Haurum 		& Søren Lyng 				\\
		&\\
		&\\
		
	\end{tabular}
\end{table}


\chapter*{Reading guide}
%Throughout the report sources are referred to by the Harvard citation method. When a source is listed in the report, the last name of the author and a publication year is listed. The sources are all listed alphabetically in the \textit{\textbf{Bibliography}} chapter. \newline
%\textit{This is an example of a source listed in the report: \textbf{\citep{safe}}.} \newline
%The source may refer to either the whole section or to only that sentence. The way this differs depends on the placement of the dot. If the dot is after the source, then that source refers to the sentence and if the dot is before the source, then the source refers to the whole section. 

%When referring to figures, tables and source code, numbers are used. Depending on which chapter and number of figure/table, the number is defined. \newline
%\textit{This example can be used: In chapter X we want to refer to the second figure. This is done by giving the figure number \textbf{X.2}, where X is the number of the chapter we're referring to.}

%When referring to source code, we're using code snippets. These code snippets aren't necessarily the full source code, but may be shorter version of it and/or missing comments. When code has been removed in the snippets, the use of three dots are used:  \textit\textbf{{"..."}}, these dots show that some code altering has been made in the snippet, whether it's because it's long and irrelevant for understanding the purpose of the code, or because we're simply just trying to explain those few lines. 

%Throughout the report requirements are split into four colours with four different meanings. The colour blue refers to new requirements or focus requirements for that specific increment. The colour green refers to a requirement being fulfilled. Orange means that the requirement is fulfilled but has been changed. Red means that the requirement has been changed, but not yet fulfilled. 
